
%%%%%%%%%%%%%%%%%%%%%%%%%%%%%%%%%%%%%%%%%%%%%%%%%%%%%%%%%%%%%%%%%%%%%%%%%
%           Capítulo 2: MARCO TEÓRICO - REVISIÓN DE LITERATURA
%%%%%%%%%%%%%%%%%%%%%%%%%%%%%%%%%%%%%%%%%%%%%%%%%%%%%%%%%%%%%%%%%%%%%%%%%

\chapter{Modelo}

En este capítulo se aborda teóricamente el modelo de bosones interactuantes
distribuidos en tres pozos, su aproximación clásica y las herramientas de
análisis dinámico aplicables a este sistema Hamiltoniano con conservación
de la energía.

\section{Modelo cuántico}

Se considera un sistema de partículas bosónicas en un potencial periódico
externo $ V(\vec{r}) = V(\vec{r} + \vec{r}_{\vec{l}}) $, donde $ \vec{r}_{\vec{l}}
    = l_{1}d_{1}\vec{e}_{1} + l_{2}d_{2}\vec{e}_{2} + l_{3}d_{3}\vec{e}_{3} $ con
$ l_{k} \in \mathbb{N} $ y $ d_{k} \in \mathbb{R} $.
El Hamiltoniano cuántico \replaced{en}{de} segunda cuantización que resulta de este potencial,
bajo el supuesto de interacciones débiles entre pares de partículas, está dado
por \added{\cite{mossmann2006semiclassical}}

\begin{equation}
    \hat{H} = \int d^{3}r \hat{\Phi}^{\dagger}(\vec{r}) \left( -\frac{\hbar^{2}}{2m}
    \Delta + V(\vec{r}) \right) \hat{\Phi}(\vec{r})
    + \frac{g}{2} \int d^{3}r \hat{\Phi}^{\dagger}(\vec{r}) \hat{\Phi}^{\dagger}(\vec{r})
    \hat{\Phi}(\vec{r}) \hat{\Phi}(\vec{r})\added{,}
    \label{eq:hamiltonianbase}
\end{equation}

\noindent donde $ m $ es la masa de las partículas y $ g $ es la constante de acoplamiento
entre ellas, la cual es positiva en el caso de interacciones repulsivas y negativa
para interacciones atractivas \cite{mossmann2006semiclassical}. A lo largo de esta tesis, se emplearán unidades
escaladas donde $\hbar = m = 1$.

El operador de campo bosónico $ \hat{\Phi}(\vec{r}) $ puede expresarse como una
expansión en términos de operadores de aniquilación:

\begin{equation}
    \hat{\Phi}(\vec{r}) = \sum_{n,m} \phi_{n,m}(\vec{r}) \hat{a}_{n,m}.
    \label{eq:field_expansion}
\end{equation}

\noindent Aquí, el índice $ n $ identifica el pozo potencial \deleted{(de entre tres)}, mientras que $ m $
se refiere a los estados excitados de cada pozo. Dado que se considera un condensado
de Bose-Einstein a temperaturas suficientemente bajas, se asume que sólo se ocupa el
estado fundamental ($ m = 1 $) \cite{mossmann2006semiclassical}.

Al sustituir la expansión \eqref{eq:field_expansion} en \eqref{eq:hamiltonianbase} y
despreciar los términos de cuarto orden en operadores con diferentes modos, se obtiene
el Hamiltoniano de Bose-Hubbard para tres pozos acoplados. A diferencia del modelo lineal
tratado en \cite{mossmann2006semiclassical}, aquí se considera una geometría anular,
donde el primer y el último pozo están también acoplados.

\begin{equation}
    \hat{H} = \hat{H}_{0} + \hat{W},
    \label{eq:hamiltonian_total}
\end{equation}

\noindent donde

\begin{equation}
    \begin{split}
        \hat{H}_0 & = \omega_1 \frac{\hat{a}^{\dagger}_{1}\hat{a}_{1} + \hat{a}_{1}\hat{a}^{\dagger}_{1}}{2}
        + \omega_2 \frac{\hat{a}^{\dagger}_{2}\hat{a}_{2} + \hat{a}_{2}\hat{a}^{\dagger}_{2}}{2}
        + \omega_3 \frac{\hat{a}^{\dagger}_{3}\hat{a}_{3} + \hat{a}_{3}\hat{a}^{\dagger}_{3}}{2}               \\
                  & + x_1 \left( \frac{\hat{a}^{\dagger}_{1}\hat{a}_{1} + \hat{a}_{1}\hat{a}^{\dagger}_{1}}{2}
        \right)^2 + x_2 \left( \frac{\hat{a}^{\dagger}_{2}\hat{a}_{2} + \hat{a}_{2}\hat{a}^{\dagger}_{2}}{2}
        \right)^2                                                                                              \\
                  & + x_3 \left( \frac{\hat{a}^{\dagger}_{3}\hat{a}_{3} + \hat{a}_{3}\hat{a}^{\dagger}_{3}}{2}
        \right)^2,
    \end{split}
    \label{eq:H0_local}
\end{equation}

\begin{equation}
    \hat{W} = -\frac{k_{12}}{2} (\hat{a}^{\dagger}_{1}\hat{a}_{2} + \hat{a}^{\dagger}_{2}\hat{a}_{1})
    - \frac{k_{23}}{2} (\hat{a}^{\dagger}_{2}\hat{a}_{3} + \hat{a}^{\dagger}_{3}\hat{a}_{2}) -
    \frac{k_{31}}{2} (\hat{a}^{\dagger}_{3}\hat{a}_{1} + \hat{a}^{\dagger}_{1}\hat{a}_{3}).
    \label{eq:W_interaction}
\end{equation}

\added{En estas expresiones, $ \hat{a}_{k} $ y $ \hat{a}^{\dagger}_{k} $ son los operadores de aniquilación
    y creación en el pozo $ k \in \{1,2,3\} $, respectivamente, que satisfacen las relaciones de conmutación bosónicas
    $ [\hat{a}_{k}, \hat{a}^{\dagger}_{l}] = \delta_{kl} $ y $ [\hat{a}_{k}, \hat{a}_{l}] = 0 $. Los parámetros $ \omega_{k} $
    representan las frecuencias de oscilación en cada pozo, $ x_{k} $ son las no linealidades debidas a
    las interacciones entre partículas en el mismo pozo, y $ k_{kl} $ son las constantes de acoplamiento entre
    los pozos $ k $ y $ l $.}

\added{Siguiendo las reglas de conmutación, se obtiene que la parte integrable $ \hat{H}_0 $ del Hamiltoniano
    \eqref{eq:hamiltonian_total} corresponde a una matriz diagonal mientras que la parte de interacción
    $ \hat{W} $ contiene los términos de acoplamiento entre pozos no diagonales \cite{mossmann2006semiclassical} .}

\added{Para simplificar el parámetro de interacciones, se asume que las no linealidades son iguales en
    cada pozo siguiendo la idea de \cite{mossmann2006semiclassical}, es decir, $ x_1 = x_2 = x_3 = x $, que permite
    una comparación más directa con resultados experimentales y manteniendo las colisiones entre bosones independientes
    del pozo.}

\deleted{ Este Hamiltoniano descarta un término constante en la energía para facilitar el análisis
    posterior. Además, se considera que los potenciales de cada pozo son iguales, es decir,
    $ x_1 = x_2 = x_3 = x $, lo cual permite una comparación más directa con resultados
    experimentales \cite{mossmann2006semiclassical}.}

\deleted{Finalmente, se introduce el operador número total de partículas $ \hat{N} = \hat{n}_1 +
        \hat{n}_2 + \hat{n}_3 $, el cual conmuta con el Hamiltoniano total \eqref{eq:hamiltonian_total},
    asegurando la conservación del número de partículas en el sistema \cite{mossmann2006semiclassical}. }

\section{Aproximación clásica}

Para realizar un análisis clásico del sistema cuántico, se construye la aproximación
clásica del Hamiltoniano mediante las reglas de sustitución de Heisenberg:

\begin{equation}
    \hat{a}_k \rightarrow \sqrt{I_k} e^{i \varphi_k}, \quad \hat{a}_k^{\dagger} \rightarrow \sqrt{I_k} e^{-i \varphi_k},
    \label{eq:heisenberg_subs}
\end{equation}

\noindent donde la variable de acción clásica $ I_k $ (con unidades de $\hbar$) se relaciona con el
número cuántico $n$ mediante

\begin{equation}
    I \leftrightarrow n + \frac{1}{2}.
    \label{eq:action_quantum_number}
\end{equation}

Los métodos semiclásicos suelen ofrecer resultados precisos en los dos primeros órdenes de
$\hbar$ (órdenes 0 y 1), con errores del orden de $\hbar^2$ \cite{mossmann2006semiclassical}.
Al aplicar las sustituciones en el Hamiltoniano \eqref{eq:hamiltonian_total} y desarrollar los
términos exponenciales, se obtiene el Hamiltoniano clásico:

\begin{equation}
    \begin{aligned}
        H(\varphi_1, \varphi_2, \varphi_3, I_1, I_2, I_3) & = \omega_1 I_1 + \omega_2 I_2 + \omega_3 I_3
        + x_1 I_1^2 + x_2 I_2^2 + x_3 I_3^2                                                                      \\
                                                          & - k_{12} \sqrt{I_1 I_2} \cos(\varphi_1 - \varphi_2)  \\
                                                          & - k_{23} \sqrt{I_2 I_3} \cos(\varphi_2 - \varphi_3)  \\
                                                          & - k_{31} \sqrt{I_3 I_1} \cos(\varphi_3 - \varphi_1).
    \end{aligned}
    \label{eq:hamiltonian_classical}
\end{equation}

Este Hamiltoniano describe un sistema clásico de tres osciladores anharmónicos acoplados, expresado
en coordenadas de acción-ángulo, con $\varphi_k \in [0, 2\pi)$, $I_k > 0$ \added{e $I_k < N$}. La parte integrable $H_0$
del sistema (sin términos de acoplamiento) conserva las acciones, mientras que los términos de
interacción $W$ las modifican al depender de las diferencias angulares y acoplamientos entre osciladores.

El corchete de Poisson entre $H$ y el observable

\begin{equation}
    K = I_{1} + I_{2} + I_{3}
    \label{eq:N_constant}
\end{equation}

\noindent es igual a cero, lo que corresponde con su análogo cuántico: la conservación del número total de partículas
\cite{mossmann2006semiclassical}. Nótese que, de acuerdo con la definición \eqref{eq:action_quantum_number},
se tiene que $K = N + 3/2$. La simetría $\{H, K\} = 0$ permite reducir el número de grados de libertad de
tres a dos mediante una transformación canónica. Utilizando la función generadora

\begin{equation}
    G(\varphi_{1}, \varphi_{2}, \varphi_{3}, J_{1}, J_{2}, K) = J_{1}(\varphi_{1} - \varphi_{2}) +
    J_{2}(\varphi_{3} - \varphi_{2}) + K \varphi_{2},
    \label{eq:generating_function}
\end{equation}

\noindent con las nuevas acciones ($J_{1}, J_{2}, K$) se obtienen las siguientes transformaciones:

\begin{equation}
    \begin{split}
        \psi_{1} & = \varphi_{1} - \varphi_{2}, \quad \psi_{2} = \varphi_{3} - \varphi_{2},
        \quad \vartheta = \varphi_{2},                                                      \\
        I_{1}    & = J_{1}, \quad I_{3} = J_{2},
    \end{split}
    \label{eq:transformation}
\end{equation}

\noindent donde ($\psi_{1}, \psi_{2}, \vartheta$) son los nuevos ángulos. El Hamiltoniano expresado
en estas nuevas coordenadas toma la forma:

\begin{equation}
    \begin{split}
        H & = \omega_1 J_1 + \omega_2 (K - J_1 - J_2) + \omega_3 J_2 \\
          & + x_1 J_1^2 + x_2 (K - J_1 - J_2)^2 + x_3 J_2^2          \\
          & - k_{12} \sqrt{J_1 (K - J_1 - J_2)} \cos\psi_1           \\
          & - k_{23} \sqrt{J_2 (K - J_1 - J_2)} \cos\psi_2           \\
          & - k_{31} \sqrt{J_1 J_2} \cos(\psi_2 - \psi_1),
    \end{split}
    \label{eq:transform_hamiltonian}
\end{equation}

\noindent observando que no hay dependencia de $\vartheta$, lo que confirma que $K$ es una constante del movimiento
como se verá más adelante en el análisis dinámico.

\section{Análisis dinámico}

Para llevar a cabo el análisis dinámico del Hamiltoniano \eqref{eq:transform_hamiltonian}, es conveniente
reducir el número de parámetros libres. Como se indicó anteriormente, se igualan los parámetros $x_k$ a un
único parámetro $x$. Además, se fijan las frecuencias como $\omega_1 = \omega$, $\omega_2 = 0$ y $\omega_3 = -\omega$,
\replaced{buscando una simetría de frecuencias respecto al pozo 2}{de modo que los pozos 1 y 3 tengan frecuencias
    de igual magnitud pero con signos opuestos}. Finalmente,
se establecen las constantes de acoplamiento $k_{12} = k_{23} = k_2$, mientras que $k_{31} = k_1$ se
mantiene como parámetro libre. Bajo estas condiciones, el Hamiltoniano se expresa como:

\begin{equation}
    \begin{split}
        H & = \omega ( J_1 - J_2 )                                                                                \\
          & + x ( J_1^2 + (K - J_1 - J_2)^2 + J_2^2 )                                                             \\
          & - k_{1} \sqrt{J_1 J_2} \cos(\psi_2 - \psi_1)                                                          \\
          & - k_{2} \left[ \sqrt{J_1 (K - J_1 - J_2)} \cos\psi_1 + \sqrt{J_2 (K - J_1 - J_2)} \cos\psi_2 \right].
    \end{split}
    \label{eq:free_param_hamiltonian}
\end{equation}

Este Hamiltoniano contiene cuatro parámetros libres. El esquema del sistema correspondiente es:

\begin{figure}[H]
    \centering
    \begin{tikzpicture}[
            every node/.style={circle, draw, minimum size=1.2cm, font=\large},
            thick
        ]

        % Coordenadas de los pozos (triángulo invertido)
        \node (1) at (-2, 2) {1};
        \node (2) at (0, 0)  {2};
        \node (3) at (2, 2)  {3};

        % Flechas dobles con etiquetas (usando latex-latex, más compatible)
        \draw[latex-latex] (1) -- (2) node[midway, left=2pt, font=\normalsize, draw=none, shape=rectangle] {$k_{2}$};
        \draw[latex-latex] (2) -- (3) node[midway, right=2pt, font=\normalsize, draw=none, shape=rectangle] {$k_{2}$};
        \draw[latex-latex] (1) -- (3) node[midway, above=2pt, font=\normalsize, draw=none, shape=rectangle] {$k_{1}$};

        % Frecuencias
        \node[draw=none, fill=none, font=\small] at ($(1)+(0,1)$)    {$\omega_1 = \omega$};
        \node[draw=none, fill=none, font=\small] at ($(2)+(0,-1)$)   {$\omega_2 = 0$};
        \node[draw=none, fill=none, font=\small] at ($(3)+(0,1)$)    {$\omega_3 = -\omega$};

        % No linealidades
        % \node[draw=none, fill=none, font=\small] at (0,-2) {Todos los $x_j = x$};

    \end{tikzpicture}

    \caption{Esquema del modelo anular con acoplamientos simétricos $k_{2}$ y uno independiente $k_{1}$.}
    \label{fig:modelo_anular}
\end{figure}

Es posible reducir un parámetro adicional dividiendo todo el Hamiltoniano entre uno de los
parámetros. A lo largo de este trabajo, se utiliza $k_2$ como unidad de referencia. Así,
se define:

\begin{equation}
    \begin{split}
        \tilde{H} & = \tilde{\omega} ( J_1 - J_2 )                                                                  \\
                  & + \tilde{x} ( J_1^2 + (K - J_1 - J_2)^2 + J_2^2 )                                               \\
                  & - \tilde{k} \sqrt{J_1 J_2} \cos(\psi_2 - \psi_1)                                                \\
                  & - \left[ \sqrt{J_1 (K - J_1 - J_2)} \cos\psi_1 + \sqrt{J_2 (K - J_1 - J_2)} \cos\psi_2 \right],
    \end{split}
    \label{eq:normalize_hamiltonian}
\end{equation}

\noindent donde $ \tilde{H} = H / k_{2} $, $ \tilde{\omega} = \omega/ k_{2} $, $ \tilde{x} = x / k_{2} $
y $ \tilde{k} = k_{1} / k_{2} $. Por las propiedades de los corchetes de Poisson, este nuevo
Hamiltoniano conserva las mismas propiedades dinámicas que el original \eqref{eq:transform_hamiltonian}.

Para llegar a las ecuaciones diferenciales que describen la dinámica de este sistema clásico, se desarrolla

\begin{equation}
    \begin{split}
        \tilde{F}(\tilde{\xi}, \tilde{p}) = \dot{\tilde{\xi}} = J\nabla_{\tilde{\xi}} H (\tilde{\xi}, \tilde{p}),
    \end{split}
    \label{eq:full_differential_system}
\end{equation}

\noindent donde $ J $ es una matriz simpléctica, $\tilde{p} = (\tilde{\omega}, \tilde{x}, \tilde{k}) \in \mathbb{R}^3$ y
$\tilde{F}: \mathbb{R}^9 \rightarrow \mathbb{R}^6$. Si $ \tilde{\xi} = (\psi_1, \psi_2, \theta, J_1, J_2, K) \in
    \mathcal{C}^1(\mathbb{R}, \mathbb{R}^6)$, se tienen las ecuaciones de Hamilton

\begin{equation}
    \begin{split}
        \dot{J}_1 & = -\frac{\partial \tilde{H}}{\partial \psi_1}, \quad
        \dot{\psi}_1 = \frac{\partial \tilde{H}}{\partial J_1}              \\
        \dot{J}_2 & = -\frac{\partial \tilde{H}}{\partial \psi_2}, \quad
        \dot{\psi}_2 = \frac{\partial \tilde{H}}{\partial J_2}              \\
        \dot{K}   & = -\frac{\partial \tilde{H}}{\partial \vartheta}, \quad
        \dot{\vartheta} = \frac{\partial \tilde{H}}{\partial K}.
    \end{split}
    \label{eq:dynamical_hamilton}
\end{equation}

Como el Hamiltoniano \eqref{eq:normalize_hamiltonian} no depende de \replaced{$ \vartheta $}{$ K $},
\replaced{$ K $}{esta variable} pasa a ser un parámetro del sistema
al no depender del tiempo \added{y $\vartheta$ es una variable cíclica}. En consecuencia, las ecuaciones
para $\dot{\vartheta}$ y $\dot{K}$ son:

\begin{equation}
    \begin{split}
        \dot{\vartheta} & = 2 \tilde{x} (K - J_{2} - J_{1})                                  \\
                        & - \frac{1}{2} \left( \sqrt{ \frac{K - J_1 - J_2}{J_1} } \cos\psi_1
        + \sqrt{\frac{K - J_1 - J_2}{J_2}} \cos\psi_2 \right),                               \\
        \dot{K}         & = 0.
    \end{split}
    \label{eq:independent_variables}
\end{equation}

Por ello, el sistema \eqref{eq:full_differential_system} puede reducirse a

\begin{equation}
    \begin{split}
        F(\xi, p) = \dot{\xi} = J\nabla_{\xi} H (\xi, p),
    \end{split}
    \label{eq:differential_system}
\end{equation}

\noindent donde $p = (\tilde{\omega}, \tilde{x}, \tilde{k}, K) \in \mathbb{R}^4$, $F: \mathbb{R}^8 \rightarrow \mathbb{R}^4$ y
$ \xi = (\psi_1, \psi_2, J_1, J_2) \in \mathcal{C}^1(\mathbb{R}, \mathbb{R}^4) $. El sistema de ecuaciones diferenciales
correspondiente es:

\begin{equation}
    \begin{split}
        \dot{\psi}_1 & = \tilde{\omega} + 2 \tilde{x} (2J_1 + J_2 - K) - \frac{\tilde{k}}{2}  \sqrt{ \frac{J_2}{J_1} } \cos(\psi_2 - \psi_1)   \\
                     & - \frac{1}{2} \left( \sqrt{ \frac{K - J_1 - J_2}{J_1} } - \sqrt{\frac{J_1}{K - J_1 - J_2}} \right) \cos\psi_1           \\
                     & + \frac{1}{2} \sqrt{\frac{J_2}{K - J_1 - J_2}} \cos\psi_2,                                                              \\
        \dot{\psi}_2 & = - \tilde{\omega} + 2 \tilde{x} (2J_2 + J_1 - K) - \frac{\tilde{k}}{2}  \sqrt{ \frac{J_1}{J_2} } \cos(\psi_2 - \psi_1) \\
                     & - \frac{1}{2} \left( \sqrt{ \frac{K - J_1 - J_2}{J_2} } - \sqrt{\frac{J_2}{K - J_1 - J_2}} \right) \cos\psi_2           \\
                     & + \frac{1}{2} \sqrt{\frac{J_1}{K - J_1 - J_2}} \cos\psi_1,                                                              \\
        \dot{J}_1    & = - \sqrt{J_1(K - J_1 - J_2)} \sin\psi_1 - \tilde{k} \sqrt{J_1J_2} \sin(\psi_2 - \psi_1),                               \\
        \dot{J}_2    & = - \sqrt{J_2(K - J_1 - J_2)} \sin\psi_2 + \tilde{k} \sqrt{J_1J_2} \sin(\psi_2 - \psi_1).
    \end{split}
    \label{eq:dynamical_system}
\end{equation}

\noindent Este sistema será el objeto principal de estudio en esta tesis.

\section{Equilibrio}

Una solución $\xi_{\text{eq}} \in \mathcal{C}^1(\mathbb{R}, \mathbb{R}^4)$ del sistema
\eqref{eq:differential_system} se denomina de equilibrio si satisface $F(\xi_{eq}, p) = \textbf{0}$.
Cabe señalar que este equilibrio no corresponde necesariamente a un equilibrio del sistema clásico completo descrito por el Hamiltoniano
\eqref{eq:hamiltonian_classical}, ya que, debido a la transformación \eqref{eq:transformation},
el equilibrio en los ángulos $\psi_i$ implica un \textbf{equilibrio relativo} de los ángulos
$\varphi_i$ (con $i \in \{1,3\}$) respecto a la evolución del ángulo independiente $\varphi_2$.

Las soluciones de equilibrio pueden clasificarse linealmente como estables, inestables o no hiperbólicas,
dependiendo del conjunto de autovalores del jacobiano evaluado en dicha solución \added{\cite{meyer1992introduction}}.
La clasificación es la siguiente:

\begin{itemize}

    \item \textbf{Estabilidad lineal:} Todas las partes reales de los autovalores del jacobiano son negativas.
          En este caso, una pequeña perturbación induce una evolución que converge al equilibrio.

    \item \textbf{Inestabilidad lineal:} Al menos un autovalor posee parte real positiva, lo que implica que
          una pequeña perturbación aleja la trayectoria del equilibrio.

    \item \textbf{No hiperbólica:} Alguno de los autovalores tiene parte real exactamente igual a cero,
          mientras que los restantes tienen parte real menor o igual que cero. En este caso, la estabilidad
          no puede determinarse únicamente mediante el análisis lineal, y es necesario considerar términos
          de orden superior.

\end{itemize}

Sin embargo, la estructura simpléctica del jacobiano del sistema \eqref{eq:differential_system} garantiza que
los autovalores siempre aparecen en pares simétricos respecto a los ejes real e imaginario. Es decir, si
$\lambda$ es un autovalor, entonces también lo son $-\lambda$, $\bar{\lambda}$ y $-\bar{\lambda}$ \cite{meyer1992introduction}.
Como consecuencia, el sistema \eqref{eq:dynamical_system} no puede presentar soluciones de equilibrio
estables en el sentido lineal, ya que la presencia de un autovalor real negativo implica necesariamente
la existencia de uno positivo. Además, la estabilidad lineal en este contexto supondría la existencia de disipación de energía,
lo cual contradice la naturaleza conservativa del sistema.

En los casos en que los autovalores son puramente reales o puramente imaginarios, se observan dos pares simétricos.
Si un autovalor tiene parte real e imaginaria distintas de cero, el conjunto completo de autovalores se obtiene
por simetría conjugada respecto a ambos ejes.

Para estudiar la estabilidad de un equilibrio con autovalores puramente imaginarios (parte real nula), es necesario
analizar el Hamiltoniano en términos de la función de Lyapunov. En tales casos, se considera que el equilibrio es estable en el
sentido de Lyapunov si el Hamiltoniano \eqref{eq:normalize_hamiltonian} posee un máximo o mínimo local en dicho
punto \added{y es inestable cuando posee un punto silla}, de acuerdo con el teorema de Dirichlet \cite{meyer1992introduction}.
\replaced{Si el Hamiltoniano presenta una matriz Hessiana singular en el equilibrio, no es posible determinar la estabilidad
    mediante los autovalores del método lineal y se requiere un análisis de orden superior para determinar la geometría del
    Hamiltoniano en el equilibrio. En términos generales, este sigue siendo un problema abierto \cite{dos2010stability}.}
{Si el Hamiltoniano presenta un punto silla o un punto
    degenerado en el equilibrio con autovalores puramente imaginarios, la estabilidad no puede determinarse de manera concluyente,
    y es necesario realizar un análisis de orden superior del Hamiltoniano. En términos generales, este sigue siendo un problema abierto \cite{dos2010stability}.}

Por tanto\added{, si se adopta como función de Lyapunov el polinomio de grado 2 \( H_2 \) de la expansión de Taylor \cite{dos2010stability},}
en este trabajo se propone la siguiente clasificación para las soluciones de equilibrio:

\begin{itemize}

    \item \textbf{Estable (en el sentido de Lyapunov):} Existen dos pares simétricos de autovalores puramente imaginarios
          (parte real cero) y \added{la función de Lyapunov} tiene un mínimo o máximo local en el equilibrio. Una pequeña
          perturbación induce oscilaciones alrededor del equilibrio, manteniendo constante la energía.

    \item \textbf{Inestable:} Existe al menos un autovalor con parte real distinta de cero. Una
          perturbación hace que el sistema se aleje del equilibrio, conservando la energía.

    \item \textbf{No concluyente:} \replaced{Se presentan autovalores puramente imaginarios y la matriz
              Hessiana de la función de Lyapunov es singular en el equilibrio. En este caso, podrían surgir inestabilidades no lineales
              asociadas a resonancias.}{Se presentan autovalores puramente imaginarios, pero el Hamiltoniano
              no exhibe un máximo ni un mínimo local. En este caso, podrían surgir inestabilidades no lineales
              asociadas a resonancias.}

\end{itemize}

\subsection{Bifurcaciones de equilibrio}

Cuando una solución de equilibrio presenta un autovalor cuya parte real e imaginaria son nulas, se
dice que ocurre una bifurcación. A lo largo de este trabajo se considerarán dos tipos de bifurcaciones
de equilibrio: el \textbf{punto límite} y el \textbf{punto de ramificación}.

Otra bifurcación de interés es la bifurcación de Hopf; sin embargo, en sistemas Hamiltonianos su análisis es
considerablemente más complejo, debido a la simetría de los autovalores y a las posibles inestabilidades inducidas
por resonancias \added{\cite{lahiri2001hamiltonian}}. Por esta razón, dichas soluciones no serán abordadas en el presente trabajo.

\begin{itemize}
    \item \textbf{Punto límite: }También conocida como \textbf{saddle-node}, esta bifurcación se caracteriza por una solución de
          equilibrio cuyo jacobiano presenta un defecto de rango igual a 1; es decir, su núcleo (espacio nulo)
          tiene dimensión 1 \cite{govaertsbifurcations}.

          Es importante destacar que, debido a la estructura simpléctica del sistema, si un autovalor
          es cero, otro autovalor también debe serlo por simetría. No obstante, ambos comparten el mismo
          autovector \cite{meyer1992introduction}. Por tanto, si el otro par de autovalores no es nulo, el espacio nulo del jacobiano
          sigue siendo unidimensional.

          Esto implica que, aunque el autovalor cero tenga multiplicidad algebraica 2, el autovector correspondiente
          es único. Esta situación permite la existencia de bifurcaciones tipo punto límite.

    \item \textbf{Punto de ramificación: }Esta bifurcación ocurre cuando el jacobiano de una solución de equilibrio presenta un defecto de
          rango igual a 2; es decir, su espacio nulo es bidimensional \cite{govaertsbifurcations}.

          Siguiendo la misma lógica que en el caso del punto límite, esta situación se da cuando existen dos
          pares de autovalores nulos, cada uno con su respectivo autovector independiente. En otras palabras,
          el núcleo del jacobiano tiene dimensión 2.

\end{itemize}



\section{Órbitas periódicas}

Una órbita periódica de periodo $T > 0$ del sistema de ecuaciones \eqref{eq:differential_system} es una solución
$\xi \in \mathcal{C}^1([0,T], \mathbb{R}^4)$ que satisface:

\begin{equation}
    \dot{\xi}(t) = F(\xi(t), p), \quad \xi(0) = \xi(T), \quad \forall t \in [0,T].
    \label{eq:orbit_mean}
\end{equation}

Al igual que las soluciones de equilibrio, las órbitas periódicas pueden clasificarse según su estabilidad.
La herramienta matemática central para este análisis es la \textbf{teoría de Floquet} \cite{meyer1992introduction}, la cual considera
perturbaciones infinitesimales a lo largo de la órbita periódica. Para ello, se introduce una función de
variación $v(t) \in \mathcal{C}^1([0,T], \mathbb{R}^4)$ tal que la condición inicial perturbada es:

\begin{equation}
    \tilde{\xi}(0) = \xi(0) + \epsilon v(0),
    \label{eq:variationalorbit}
\end{equation}

\noindent y al sustituir esta condición en \eqref{eq:orbit_mean} se obtiene la ecuación variacional para $v(t)$:

\begin{equation}
    \begin{split}
        \dot{v}(t)              & = F_{\xi}(\xi(t), p)\, v(t),                                              \\
        F_{\xi}(\xi(t), p)_{ij} & = \left. \dfrac{\partial F_{i}}{\partial \xi_{j}} \right|_{\xi = \xi(t)}.
    \end{split}
    \label{eq:variationaequation}
\end{equation}

La solución de esta ecuación define un mapeo de evolución:

\begin{equation}
    v(T) = M v(0),
    \label{eq:floquetmatrix}
\end{equation}

\noindent donde $M$ es la \textbf{matriz de Floquet}. Los autovalores de $M$ se denominan \textbf{multiplicadores de Floquet}
y caracterizan la estabilidad de la órbita periódica. Todo sistema autónomo posee al menos un
multiplicador igual a 1, correspondiente a la dirección tangente a la órbita.

La clasificación de estabilidad es la siguiente:

\begin{itemize}
    \item \textbf{Estable:} Todos los multiplicadores de Floquet se encuentran dentro del círculo
          unitario complejo. Si hay multiplicadores sobre el círculo, deben ser \emph{simples}
          (es decir, con multiplicidad algebraica igual a su multiplicidad geométrica). En este caso,
          pequeñas perturbaciones no provocan que la trayectoria se aleje de la órbita.

    \item \textbf{Inestable:} Al menos un multiplicador se encuentra fuera del círculo unitario
          complejo, lo que implica que una pequeña perturbación se amplifica con el tiempo.
\end{itemize}

En el contexto Hamiltoniano, se deben considerar propiedades adicionales. La conservación de la
energía implica que siempre existen al menos dos multiplicadores unidad sobre el círculo unitario.
A diferencia del caso de equilibrio, aquí los dos autovalores unidad están asociados a autovectores
\emph{linealmente independientes}, lo cual refleja la existencia de una familia continua de órbitas
periódicas cercanas.

Los dos multiplicadores restantes aparecen con simetría respecto a la inversión y conjugación compleja:
si $\mu$ es un multiplicador distinto de 1, entonces el conjunto completo es $\{1, 1, \mu, 1/\mu, \bar{\mu}, 1/\bar{\mu}\}$.
Como máximo en el sistema \eqref{eq:differential_system} puede haber 4 autovalores, entonces con la simetría del conjunto
se deduce que los autovalores con parte imaginaria no nula siempre permanecen sobre el circulo unitario complejo
\cite{meyer1992introduction}.

En la Figura \ref{fig:floquet_multipliers} se muestra la distribución típica de los multiplicadores de Floquet en el
plano complejo para una órbita periódica en un sistema Hamiltoniano conservativo.

\begin{figure}[H]
    \centering
    \begin{tikzpicture}[scale=3]
        % Círculo unitario
        \draw[thick] (0,0) circle (1);

        % Ejes
        \draw[->] (-1.5,0) -- (1.5,0) node[right] {\small $\text{Re}(z)$};
        \draw[->] (0,-1.5) -- (0,1.5) node[above] {\small $\text{Im}(z)$};

        % Etiqueta del círculo
        \node at (0.5,0.5) {\small $|z|=1$};

        % Puntos en +1 y -1
        \fill[red] (1,0) circle (0.03) node[below right=1pt] {\small $\mu_1 = \mu_2 = 1$};
        \fill[red] (-1,0) circle (0.03);

        \fill[red] (1.2,0) circle (0.03);
        \fill[red] (0.9,0) circle (0.03);
        \fill[red] (0.2,0) circle (0.03);

        \fill[red] (-1.2,0) circle (0.03);
        \fill[red] (-0.9,0) circle (0.03);
        \fill[red] (-0.2,0) circle (0.03);


        % Otros puntos con coordenadas polares alrededor de -1
        \foreach \angle in {30, 60, 90, 120, 150, 180, 210, 240, 270, 300, 330, 360} {
                \fill[red] (\angle:1) circle (0.03);
            }

        \foreach \angle in {5, 15, -5, -15} {
                \fill[red] (\angle:1) circle (0.03);
            }

        \foreach \angle in {175, 165, 185, 195} {
                \fill[red] (\angle:1) circle (0.03);
            }

    \end{tikzpicture}
    \caption{Distribución de los multiplicadores de Floquet en el círculo unitario}
    \label{fig:floquet_multipliers}
\end{figure}

\subsection{Bifurcaciones de órbitas periódicas}

Con base en lo discutido en la sección anterior, existen dos tipos de órbitas periódicas particularmente
relevantes en el sistema que se está estudiando: aquellas en las que todos los multiplicadores de
Floquet son iguales a 1, y aquellas en las que un par de multiplicadores es igual a 1 y el otro par
es igual a $-1$. Estos dos casos están asociados a bifurcaciones específicas de órbitas periódicas:

\begin{itemize}

    \item \textbf{Ciclo \replaced{de ramificación}{límite}: } \replaced{Esta bifurcación ocurre cuando todos los multiplicadores de
              Floquet son iguales a $1$. Debido a la simetría de los multiplicadores, sólo se tienen asociados 3 autovectores
              linealmente independientes, esto es lo que permite obtener este tipo de bifurcaciones \cite{galan2007continuation}.
              La órbita inicial de esta nueva rama de órbitas periódicas que emerge de esta bifurcación no aumenta en periodo.}
          {Esta bifurcación ocurre cuando dos multiplicadores de Floquet son
              iguales a 1 y están asociados a autovectores linealmente independientes. Debido a la simetría
              espectral de los sistemas Hamiltonianos, el multiplicador $1$ aparece con multiplicidad algebraica
              4, pero con multiplicidad geométrica 3 además que se puede simplificar un multiplicador con su autovector,
              reduciendo el problema a un multiplicador con multiplicidad algebraica 3 y geometríca 2 \cite{galan2007continuation}.
              Esta degeneración permite que, aun habiendo múltiples multiplicadores iguales a 1, pueda producirse una bifurcación en ese punto.}

          \textbf{Duplicación de período: } \replaced{Esta bifurcación ocurre cuando dos multiplicadores de
              Floquet son iguales a $-1$. Nuevamente por la simetría de los multiplicadores, sólo se tiene asociado 1 autovector
              linealmente independiente a este par de autovalores \cite{galan2007continuation}.
              La órbita inicial de esta nueva rama de órbitas periódicas que emerge de esta bifurcación duplica el periodo.}{Esta bifurcación tiene lugar cuando un multiplicador de
              Floquet es igual a $-1$. Aunque en sistemas Hamiltonianos este multiplicador aparece por pares
              (es decir, hay dos valores iguales a $-1$), la estructura del espacio propio, junto con la simetría
              de los multiplicadores, permite la existencia de esta bifurcación. El resultado es una órbita
              periódica cuyo nuevo periodo es el doble del original.}
\end{itemize}





