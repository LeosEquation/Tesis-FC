
%%%%%%%%%%%%%%%%%%%%%%%%%%%%%%%%%%%%%%%%%%%%%%%%%%%%%%%%%%%%%%%%%%%%%%%%%
%           Capítulo 3: NOMBRE                   %
%%%%%%%%%%%%%%%%%%%%%%%%%%%%%%%%%%%%%%%%%%%%%%%%%%%%%%%%%%%%%%%%%%%%%%%%%

\chapter{Métodos}

En este capítulo se describen los métodos matemáticos y numéricos empleados
para analizar los sistemas dinámicos estudiados en este trabajo. Primero,
se presentan los fundamentos teóricos, incluyendo el Teorema de la Función
Implícita y la técnica de continuación por pseudo-longitud de arco. Luego,
se desarrollan los sistemas extendidos para continuar ramas de equilibrio,
detectar bifurcaciones y estudiar órbitas periódicas, incluyendo su
duplicación de periodo. Finalmente, se introduce la sección de Poincaré
como herramienta para visualizar la dinámica en el espacio de fases y
reconocer estructuras periódicas y caóticas.

\section{Teorema de la Función Implícita}

Para establecer la existencia, unicidad, parametrización y diferenciabilidad
de las soluciones de un sistema de ecuaciones no lineales, es fundamental recurrir
a resultados clásicos del análisis funcional, en particular al Teorema de la
Función Implícita y a sus extensiones \cite{Doedel2007}.

\begin{theorem}[Teorema de la Función Implícita]
    Sea $\mathbf{G} : \mathcal{B} \times \mathbb{R}^m \to \mathcal{B}$, donde
    $\mathcal{B}$ es un espacio de Banach y $m \in \mathbb{N}$, que satisface
    las siguientes condiciones:

    \begin{itemize}
        \item $ \mathbf{G}(\mathbf{u}_0, \boldsymbol{\lambda}_0) = 0 $, para
              $ \mathbf{u}_0 \in \mathcal{B} $ y $ \boldsymbol{\lambda}_0 \in \mathbb{R}^m $.

        \item $ \mathbf{G}_{\mathbf{u}}(\mathbf{u}_0, \boldsymbol{\lambda}_0) $ es
              invertible con inversa acotada, es decir,
              $ \|\mathbf{G}_{\mathbf{u}}(\mathbf{u}_0, \boldsymbol{\lambda}_0)^{-1} \| < M $
              para algún $ M > 0 $.

        \item $\mathbf{G}$ y $\mathbf{G}_{\mathbf{u}}$ son continuas de Lipschitz. En particular,
              para todo $\mathbf{u}, \mathbf{v} \in S_\rho(\mathbf{u}_0) = \{\mathbf{u} \in
                  \mathcal{B}\ |\ \|\mathbf{u}_0 - \mathbf{u}\| \leq \rho \} $ y para todo
              $\boldsymbol{\lambda}, \boldsymbol{\mu} \in S_\rho(\boldsymbol{\lambda}_0)$,
              existe una constante $K_L > 0$ tal que:
              \begin{align*}
                  \| \mathbf{G}(\mathbf{u},\boldsymbol{\lambda}) - \mathbf{G}(\mathbf{v},\boldsymbol{\mu})\| & \leq
                  K_L\big(\|\mathbf{u}-\mathbf{v}\| + \|\boldsymbol{\lambda}-\boldsymbol{\mu}\|\big),                                                   \\[6pt]
                  \| \mathbf{G}_{\mathbf{u}}(\mathbf{u},\boldsymbol{\lambda}) - \mathbf{G}_{\mathbf{u}}(\mathbf{v},
                  \boldsymbol{\mu})\|                                                                        & \leq K_L\big(\|\mathbf{u}-\mathbf{v}\| +
                  \|\boldsymbol{\lambda}-\boldsymbol{\mu}\|\big).
              \end{align*}
    \end{itemize}

    Entonces, existe $\delta$, con $0 < \delta \leq \rho$, y una función
    $\mathbf{u}(\boldsymbol{\lambda})$ única y continua en $S_{\delta}(\boldsymbol{\lambda}_0)$
    tal que \deleted{\cite{Doedel2007}}:
    \begin{itemize}
        \item $\mathbf{u}(\boldsymbol{\lambda}_0) = \mathbf{u}_0$,
        \item $\mathbf{G}(\mathbf{u}(\boldsymbol{\lambda}), \boldsymbol{\lambda}) = 0$
              para toda $ \boldsymbol{\lambda} \in S_{\delta}(\boldsymbol{\lambda}_0)$.
    \end{itemize}
    \label{th:IFT}
\end{theorem}

\begin{theorem}
    Bajo las hipótesis del Teorema de la Función Implícita, si además la
    derivada $\mathbf{G}_{\boldsymbol{\lambda}}(\mathbf{u},\boldsymbol{\lambda})$ es continua
    en $S_\rho(\mathbf{u}_0) \times S_\rho(\boldsymbol{\lambda}_0)$ \cite{Doedel2007},
    entonces la rama de soluciones $\mathbf{u}(\boldsymbol{\lambda})$ posee una derivada
    continua respecto de $\boldsymbol{\lambda}$, es decir,
    $\mathbf{u}_{\boldsymbol{\lambda}}(\boldsymbol{\lambda})$, en
    $S_\delta(\mathbf{u}_0) \times S_\delta(\boldsymbol{\lambda}_0)$.
    \label{th:Differentiation}
\end{theorem}

En este contexto, si $\mathbf{G}(\mathbf{u},\boldsymbol{\lambda}_0)=0$ representa un sistema
de ecuaciones y $\mathbf{G}_{\mathbf{u}}(\mathbf{u}_0,\boldsymbol{\lambda}_0)$ es invertible con
inversa acotada, se dice que $\mathbf{u}_0$ es una \emph{solución aislada} de
$\mathbf{G}(\mathbf{u},\boldsymbol{\lambda}_0)=0$ \cite{Doedel2007}. Bajo estas condiciones,
y suponiendo la continuidad de Lipschitz de $\mathbf{G}$, el Teorema de la Función Implícita
garantiza la existencia de una \emph{rama de soluciones}
$\mathbf{u} = \mathbf{u}(\boldsymbol{\lambda})$ definida localmente, con
$\mathbf{u}(\boldsymbol{\lambda}_0)=\mathbf{u}_0$ y única en un entorno de
$\boldsymbol{\lambda}_0$ \cite{Doedel2007}.



\section{Continuación por Pseudo-Longitud de Arco}

Sea $ \mathbf{G} : \mathbb{R}^{n+1} \to \mathbb{R}^n$ una función que cumple
las hipótesis del Teorema de la Función Implícita. \replaced{Sea}{Consideremos}
$ \mathbf{u} \in \mathbb{R}^n $ y $ \lambda \in \mathbb{R} $, con $n \in \mathbb{N}$.
En estas condiciones, la función $ \mathbf{u}(\lambda) \in \mathbb{R}^n $ existe,
es continua y además derivable respecto a $\lambda$ \cite{Doedel2007,Allgower1990}.

Para resolver el sistema $ \mathbf{G}(\mathbf{u}, \lambda) = 0 $ para un valor fijo de
$ \lambda $, se utiliza el método de Newton-Rhapson, el cual se escribe como:

\begin{equation}
    \begin{cases}
        \mathbf{G}_{\mathbf{u}}(\mathbf{u}^{(\nu)}, \lambda)
        \Delta \mathbf{u}^{(\nu)} = - \mathbf{G}(\mathbf{u}^{(\nu)}, \lambda), \\
        \mathbf{u}^{(\nu + 1)} = \mathbf{u}^{(\nu)} + \Delta \mathbf{u}^{(\nu)},
        \qquad \nu = 0, 1, 2, \dots
    \end{cases}
    \label{eq:NR_parameter}
\end{equation}

La efectividad de este método depende crucialmente de la elección del punto inicial
$ ( \mathbf{u}^{(0)}, \lambda ) $, pues de ello depende la convergencia del procedimiento.
Una estrategia común consiste en partir de una solución conocida
$ ( \mathbf{u}_0, \lambda_0 ) $ tal que $ \mathbf{G}( \mathbf{u}_0, \lambda_0 ) = 0 $.
Gracias a la derivabilidad de $ \mathbf{u}(\lambda) $, se puede obtener un punto inicial
$ \mathbf{u}^{(0)} $ suficientemente cercano a $ \mathbf{u} $ mediante una aproximación lineal:

\begin{equation}
    \mathbf{u}^{(0)} = \mathbf{u}_0 + (\lambda - \lambda_0)\left.\dfrac{d\mathbf{u}}{d\lambda}\right|_{\lambda_0}.
    \label{eq:parameter_step}
\end{equation}

La convergencia será más rápida cuanto más próximo sea $ \lambda $ a $ \lambda_0 $.
Una vez obtenida la solución mediante \eqref{eq:NR_parameter}, esta pasa a convertirse
en el nuevo punto de partida en \eqref{eq:parameter_step}, permitiendo así calcular
sucesivamente nuevas soluciones de la \emph{rama de soluciones}.
Este procedimiento recibe el nombre de \emph{continuación de parámetro} \cite{Doedel2007,Keller1977}.

Reescribiendo la notación, sea $\mathbf{x} = ( \mathbf{u}, \lambda )
    \in \mathbb{R}^{n + 1} $. Una solución $\mathbf{x}_0$ de $\mathbf{G}(\mathbf{x}) = 0$
se denomina \emph{regular} si la matriz Jacobiana $ \mathbf{G}_{\mathbf{x}}(\mathbf{x}_0) $,
de tamaño $n \times (n+1)$, tiene rango máximo. Es decir, si
\[
    \mathrm{Rank}(\mathbf{G}_{\mathbf{x}}(\mathbf{x}_0)) = n
    \quad \text{y} \quad
    \mathrm{Nullity}(\mathbf{G}_{\mathbf{x}}(\mathbf{x}_0)) = 1.
\]

Como se establece en \cite{Doedel2007}, se cumple el siguiente resultado:

\begin{theorem}
    Sea $\mathbf{x}_0 \equiv (\mathbf{u}_0, \lambda_0)$ una solución regular
    de $\mathbf{G}(\mathbf{x}) = 0$. Entonces, en un entorno de
    $\mathbf{x}_0$, existe un continuo unidimensional de soluciones
    $\mathbf{x}(s)$, denominado \emph{familia de soluciones} o \emph{rama de soluciones},
    tal que $\mathbf{x}(0) = \mathbf{x}_0$.
    \label{th:RegularSolution}
\end{theorem}

El teorema anterior garantiza que, aunque el Jacobiano
$ \mathbf{G}_{\mathbf{u}}(\mathbf{x}_0) $ pueda volverse singular,
en una solución regular sigue existiendo una parametrización en el sentido del
Teorema de la Función Implícita. Esto se logra reemplazando una de las columnas linealmente
dependientes del Jacobiano por la columna $ \mathbf{G}_{\lambda}(\mathbf{x}_0) $,
lo que sugiere que el parámetro de la rama de soluciones puede pasar a ser una de las
coordenadas de $\mathbf{u}_0$ en lugar de $ \lambda_0 $.

Este resultado es fundamental, pues permite inducir una parametrización más general
de $ \mathbf{x}(s) = ( \mathbf{u}(s), \lambda(s)) $, con
$ \mathbf{G}(\mathbf{x}(s)) = 0 $ y $\mathbf{x}(0) = \mathbf{x}_0$ solución regular.
Además, su derivada $\mathbf{x}' (0) = \tfrac{d\mathbf{x}_0}{ds}$
existe y pertenece al espacio nulo de $ \mathbf{G}_{\mathbf{x}}(\mathbf{x}_0)$, es decir,
\begin{equation}
    \mathbf{G}'(0) = \mathbf{G}_{\mathbf{x}}(\mathbf{x}) \mathbf{x}'(0) = 0.
    \label{eq:tangent_direction}
\end{equation}

Para llevar a cabo la continuación de soluciones regulares se utiliza el método de
\emph{continuación por pseudo-longitud de arco} propuesto por Keller \cite{Keller1977}.
\replaced{Si}{Supongamos que} se dispone de una solución regular $(\mathbf{u}_0, \lambda_0)$ de
$\mathbf{G}(\mathbf{u}, \lambda) = 0$, junto con el vector tangente
$(\mathbf{u}'_0, \lambda'_0)$ obtenido del espacio nulo en dicha solución.
La continuación por pseudo-longitud de arco plantea el siguiente sistema para
$(\mathbf{u}_1, \lambda_1)$:
\begin{equation}
    \begin{cases}
        \mathbf{G}(\mathbf{u}_1, \lambda_1) = 0, \\
        (\mathbf{u}_1 - \mathbf{u}_0)^T \mathbf{u}'_0 +
        (\lambda_1 - \lambda_0) \lambda'_0
        - \Delta s = 0,
    \end{cases}
    \label{eq:System_PALC}
\end{equation}
donde $\Delta s$ representa el paso en la parametrización a lo largo de la rama de soluciones.

El método de Newton-Rhapson para resolver este sistema se escribe como:
\begin{equation}
    \begin{pmatrix}
        \left(\mathbf{G}_{\mathbf{u}}^1\right)^{(\nu)} & \left(\mathbf{G}_{\lambda}^1\right)^{(\nu)} \\[6pt]
        \mathbf{u}_0'^{\ T}                            & \lambda'_0
    \end{pmatrix}
    \begin{pmatrix}
        \Delta \mathbf{u}_1^{(\nu)} \\[6pt]
        \Delta \lambda_1^{(\nu)}
    \end{pmatrix}
    = -
    \begin{pmatrix}
        \mathbf{G}(\mathbf{u}_1^{(\nu)}, \lambda_1^{(\nu)}) \\[6pt]
        (\mathbf{u}_1^{(\nu)} - \mathbf{u}_0)^T \mathbf{u}'_0 +
        (\lambda_1^{(\nu)} - \lambda_0) \lambda'_0
        - \Delta s
    \end{pmatrix},
    \label{eq:NR_PALC}
\end{equation}
donde
\begin{equation}
    \begin{pmatrix}
        \mathbf{u}_1^{(0)} \\[6pt]
        \lambda_1^{(0)}
    \end{pmatrix} =
    \begin{pmatrix}
        \mathbf{u}_0 \\[6pt]
        \lambda_0
    \end{pmatrix}
    + \Delta s
    \begin{pmatrix}
        \mathbf{u}'_0 \\[6pt]
        \lambda'_0
    \end{pmatrix}.
    \label{eq:first_step_PALC}
\end{equation}


Con el fin de evitar el recálculo del espacio nulo del Jacobiano en cada iteración,
se actualizan las direcciones resolviendo:
\begin{equation}
    \begin{pmatrix}
        \mathbf{G}_{\mathbf{u}}^1 & \mathbf{G}_{\lambda}^1 \\[6pt]
        \mathbf{u}_0'^{\ T}       & \lambda'_0
    \end{pmatrix}
    \begin{pmatrix}
        \mathbf{u}'_1 \\[6pt]
        \lambda'_1
    \end{pmatrix}
    =
    \begin{pmatrix}
        0 \\[6pt] 1
    \end{pmatrix},
    \label{eq:new_directions}
\end{equation}
y posteriormente se reescala cada nueva dirección de manera que
$ \lVert \mathbf{u}'_1 \rVert^2 + \lVert \lambda'_1 \rVert^2 = 1 $ \cite{Doedel2007}.


\begin{algorithm}[H]
    \caption{Continuación por Pseudo-Longitud de Arco (PALC)}
    \KwIn{Solución regular inicial $(\mathbf{u}_0, \lambda_0)$, vector tangente $(\dot{\mathbf{u}}_0, \dot{\lambda}_0)$, paso $\Delta s$}
    \KwOut{Rama de soluciones}

    \textbf{Paso 1: Predicción} \\
    Calcular el punto inicial para el paso de continuación:
    \[
        \begin{pmatrix} \mathbf{u}_1^{(0)} \\ \lambda_1^{(0)} \end{pmatrix} =
        \begin{pmatrix} \mathbf{u}_0 \\ \lambda_0 \end{pmatrix}
        + \Delta s
        \begin{pmatrix} \dot{\mathbf{u}}_0 \\ \dot{\lambda}_0 \end{pmatrix}.
    \]

    \textbf{Paso 2: Corrección} \\
    Resolver el sistema extendido usando Newton-Raphson:
    \[
        \begin{pmatrix}
            \left(\mathbf{G}_{\mathbf{u}}^1\right)^{(\nu)} & \left(\mathbf{G}_{\lambda}^1\right)^{(\nu)} \\
            \dot{\mathbf{u}}_0^T                           & \dot{\lambda}_0
        \end{pmatrix}
        \begin{pmatrix}
            \Delta \mathbf{u}_1^{(\nu)} \\[6pt]
            \Delta \lambda_1^{(\nu)}
        \end{pmatrix}
        = -
        \begin{pmatrix}
            \mathbf{G}(\mathbf{u}_1^{(\nu)}, \lambda_1^{(\nu)}) \\[6pt]
            (\mathbf{u}_1^{(\nu)} - \mathbf{u}_0)^T \dot{\mathbf{u}}_0 +
            (\lambda_1^{(\nu)} - \lambda_0) \dot{\lambda}_0
            - \Delta s
        \end{pmatrix}.
    \]

    \textbf{Paso 3 Tangente} \\
    Calcular y normalizar nueva dirección tangente:
    \[
        \begin{pmatrix}
            \mathbf{G}_{\mathbf{u}}^1 & \mathbf{G}_{\lambda}^1 \\
            \dot{\mathbf{u}}_0^T      & \dot{\lambda}_0
        \end{pmatrix}
        \begin{pmatrix}
            \dot{\mathbf{u}}_1 \\ \dot{\lambda}_1
        \end{pmatrix}
        =
        \begin{pmatrix} 0 \\ 1 \end{pmatrix}, \quad
        ||\dot{\mathbf{u}}_1||^2 + ||\dot{\lambda}_1||^2 = 1
    \]

    \textbf{Paso 4 Repetición} \\
    Guardar el valor de $ (\mathbf{u}_1, \lambda_1) $, sustituir $ \mathbf{u}_0 \to \mathbf{u}_1 $
    y $ \lambda_0 \to \lambda_1 $ y repetir los pasos 1–4 para continuar la rama de soluciones.
    \label{al:PALC}
\end{algorithm}

\section{Ramas de equilibrio}

Para estudiar una \emph{rama de equilibrio} del sistema \eqref{eq:differential_system},
se debe fijar un parámetro de continuación respecto del cual dependa
$\xi$, de modo que se cumpla
\[
    F(\xi(\lambda), \lambda) = 0,
\]
en un entorno de una solución de equilibrio \emph{regular}.
En este caso, el parámetro de interés es $\tilde{k}$.

Dado que la función $F$ satisface las hipótesis de los teoremas
\ref{th:IFT}, \ref{th:Differentiation} y \ref{th:RegularSolution} para valores de $\psi_i \in (-\pi,\pi]$ y
\[
    (J_1,J_2) \in \left\{ (J_1, J_2) \in \mathbb{R}^2 \,\middle|\,
    J_1 > 0,\; J_2 > 0,\; J_1 + J_2 < K \right\},
\]
se sigue que, en una solución de equilibrio regular
$(\xi_0, \tilde{k}_0)$ con parámetros fijos $(\tilde{\omega}, \tilde{x}, K)$,
existe una rama de equilibrio que puede calcularse mediante
un procedimiento de continuación, implementando el
Algoritmo~\ref{al:PALC}.

\subsection{Sistema extendido para soluciones de equilibrio}

El sistema extendido asociado a la continuación de soluciones de equilibrio está dado por
\begin{equation}
    \begin{cases}
        F(\xi_1, \tilde{k}_1;\, \tilde{\omega}, \tilde{x}, K) = 0, \\[6pt]
        (\xi_1 - \xi_0)^T \xi'_0 +
        (\tilde{k}_1 - \tilde{k}_0)\tilde{k}'_0
        - \Delta s = 0,
    \end{cases}
    \label{eq:System_equilibrium}
\end{equation}
donde $\Delta s$ representa el paso de pseudo-longitud de arco en la parametrización de la rama \cite{Doedel2007}.

\subsection{Sistema extendido para puntos límite}

Para detectar puntos límite a lo largo de una rama de soluciones de equilibrio,
es necesario identificar un cambio de signo en cada paso de la continuación.
Esto se logra mediante la función $ \phi_{LP} $, definida como:
\begin{equation}
    \phi_{LP}(i) = \tilde{k}'_i,
    \label{LPtest}
\end{equation}
que indica la proximidad a un punto límite \cite{Meijer2011}. Una vez localizado uno de los dos
puntos donde ocurre el cambio de signo, se toma como punto inicial $ ( \xi_{LP}^{(0)},
    \tilde{k}_{LP}^{(0)} ) $ para resolver el sistema extendido mediante el método de Newton-Raphson:
\begin{equation}
    \begin{cases}
        F(\xi_{LP}, \tilde{k}_{LP}; \tilde{\omega}, \tilde{x}, K) = 0,            \\[2mm]
        F_{\xi}(\xi_{LP}, \tilde{k}_{LP}; \tilde{\omega}, \tilde{x}, K) \, v = 0, \\[1mm]
        v \cdot v = 1,
    \end{cases}
    \label{eq:System_lp}
\end{equation}
donde se incluye el autovector $v$ del Jacobiano en $ ( \xi_{LP}, \tilde{k}_{LP} ) $
como variable independiente, inicializando con el autovector $v^{(0)}$ correspondiente
al autovalor cuya parte real está más cercana a cero en $ ( \xi_{LP}^{(0)}, \tilde{k}_{LP}^{(0)} ) $ \cite{Meijer2011}.

\subsection{Sistema extendido para puntos de ramificación}

De manera análoga a los puntos límite, un punto de ramificación se identifica mediante
un cambio de signo de una función de prueba $ \phi_{BP} $, definida como
\begin{equation}
    \phi_{BP} = \det \left( \begin{bmatrix} F_{(\xi, \tilde{k})}(\xi_i, \tilde{k}_i; \tilde{\omega}, \tilde{x}, K) \\
            ( \xi' , \tilde{k}' )^{T}
        \end{bmatrix} \right),
    \label{BPtest}
\end{equation}
que indica la cercanía a un punto de ramificación \cite{Meijer2011}. Una vez detectado un cambio de
signo, se toma como punto inicial $ ( \xi_{BP}^{(0)}, \tilde{k}_{BP}^{(0)} ) $ para
resolver el sistema extendido mediante Newton-Raphson:

\begin{equation}
    \begin{cases}
        F(\xi_{BP}, \tilde{k}_{BP}; \tilde{\omega}, \tilde{x}, K) + b \psi = 0,             \\[1mm]
        F_{\xi}(\xi_{BP}, \tilde{k}_{BP}; \tilde{\omega}, \tilde{x}, K)^{T} \psi = 0,       \\[1mm]
        F_{\tilde{k}}(\xi_{BP}, \tilde{k}_{BP}; \tilde{\omega}, \tilde{x}, K)^{T} \psi = 0, \\[1mm]
        \psi \cdot \psi = 1,
    \end{cases}
    \label{eq:System_bp}
\end{equation}

donde se incluye un parámetro de perturbación $ b $ que empieza en $ b^{(0)} = 0 $
y el autovector $ \psi $ del Jacobiano transpuesto como variable independiente,
inicializando con el autovector $ \psi^{(0)} $ correspondiente al autovalor del
Jacobiano transpuesto cuya parte real es más cercana a cero en $ ( \xi_{BP}^{(0)}, \tilde{k}_{BP}^{(0)} ) $  \cite{Meijer2011}.

\subsection{Cambio de rama de equilibrio}

En un punto de ramificación, para continuar por una rama distinta a la principal es
necesario determinar la dirección de la rama emergente. En este caso ya no es suficiente
tomar un vector del espacio nulo, pues al ser un punto no regular dicho espacio nulo es
de dimensión mayor y contiene múltiples direcciones posibles.

Para caracterizar la dirección de la rama
\((\dot{\xi}_1, \dot{\tilde{k}}_1)\), emergente a partir de la rama de equilibrio inicial,
es necesario distinguir dos situaciones \cite{Meijer2011}:

\begin{enumerate}
    \item Si $\dim \mathcal{N}\big(F_\xi(\xi_{BP}, \tilde{k}_{BP}; \tilde{\omega}, \tilde{x}, K)\big) = 1$, entonces
          \[
              \xi'_0 = v_0, \quad \tilde{k}'_0 = 0, \quad
              \xi'_1 = v_1, \quad \tilde{k}'_1 = q_1,
          \]
          donde $v_0$ genera el espacio nulo y $(v_1,q_1)$ se obtiene resolviendo
          \begin{equation}
              \begin{cases}
                  F_\xi(\xi_{BP}, \tilde{k}_{BP}; \tilde{\omega}, \tilde{x}, K)\, v_1
                  + F_{\tilde{k}}(\xi_{BP}, \tilde{k}_{BP}; \tilde{\omega}, \tilde{x}, K)\, q_1 = 0, \\[6pt]
                  v_0^T v_1 = 0,
              \end{cases}
          \end{equation}
          con la normalización $v_1^T v_1 + q_1^2 = 1$.

    \item Si $\dim \mathcal{N}\big(F_\xi(\xi_{BP}, \tilde{k}_{BP}; \tilde{\omega}, \tilde{x}, K)\big) = 2$, entonces
          \[
              \xi'_0 = v_0, \quad \tilde{k}'_0 = 0, \quad
              \xi'_1 = v_1, \quad \tilde{k}'_1 = 0,
          \]
          con $v_0, v_1$ vectores que generan el espacio nulo.
\end{enumerate}

\subsection{Continuación de bifurcaciones}

Los sistemas \eqref{eq:System_lp} y \eqref{eq:System_bp} satisfacen también las condiciones de los
teoremas \ref{th:IFT}, \ref{th:Differentiation} y \ref{th:RegularSolution}, por lo que son candidatos a continuación de soluciones. Para ello,
se agrega la ecuación de proyección
\begin{equation}
    (\xi_1 - \xi_0)^T \xi'_0 +
    (v_1 - v_0)^T v'_0 +
    (\tilde{k}_1 - \tilde{k}_0)\tilde{k}'_0 +
    (\lambda_1 - \lambda_0)\lambda'_0
    - \Delta s = 0,
\end{equation}
\begin{equation}
    (\xi_1 - \xi_0)^T \xi'_0 +
    (\psi_1 - \psi_0)^T \psi'_0 +
    (\tilde{k}_1 - \tilde{k}_0)\tilde{k}'_0 +
    (\lambda_1 - \lambda_0)\lambda'_0
    - \Delta s = 0,
\end{equation}
donde $\lambda$ representa un segundo parámetro del sistema (en este caso, $\tilde{\omega}$ o $\tilde{x}$).
Con esta formulación es posible implementar el algoritmo \ref{al:PALC} para realizar la continuación \cite{Meijer2011}.

\section{Ramas de órbitas periódicas}

Para estudiar una \emph{rama de órbitas periódicas} del sistema \eqref{eq:differential_system},
se introduce un parámetro de continuación respecto del cual dependen tanto
$\xi$ como el periodo $T$. De esta forma, el problema se formula como
\begin{equation}
    \dfrac{d\xi}{dt} = F(\xi(t, \lambda), \lambda ; \tilde{\omega}, \tilde{x}, \tilde{k}, K), \quad
    \xi(0) = \xi(T(\lambda), \lambda), \quad \forall t \in [0,T(\lambda)].
    \label{eq:orbit_param}
\end{equation}

A diferencia de una solución de equilibrio, el sistema ahora considera como variables
la trayectoria periódica $ \xi(\cdot) $, el periodo $ T > 0 $ y el parámetro de continuación $\lambda$.
Con el fin de normalizar el intervalo temporal, se realiza el cambio de variable \cite{Doedel2007}
\begin{equation}
    \dfrac{d\xi}{dt} = T(\lambda)\, F(\xi(t, \lambda), \lambda ; \tilde{\omega}, \tilde{x}, \tilde{k}, K), \quad
    \xi(0) = \xi(1, \lambda), \quad \forall t \in [0,1].
    \label{eq:orbit_one}
\end{equation}


\subsection{Sistema extendido para soluciones de órbitas periódicas en sistemas conservativos}

El sistema extendido para la continuación de órbitas periódicas se plantea como
\begin{equation}
    \begin{cases}
        \dfrac{d\xi}{dt} = T F(\xi, \lambda; \tilde{\omega}, \tilde{x}, \tilde{k}, K), \\[1mm]
        \xi(0) = \xi(1),                                                               \\[1mm]
        \int_0^1 \xi(t)^{T} \dot{\xi}_0(t) \, dt = 0,                                  \\[1mm]
        \int_0^1 (\xi(t) - \xi_0(t))^{T} \xi'_0(t)\, dt
        + (T - T_0) T_0' + (\lambda - \lambda_0)\lambda_0' = \Delta s ,
    \end{cases}
    \label{eq:orbit_sistem}
\end{equation}
donde $ \xi_0(\cdot) $ es una órbita periódica de referencia con periodo $T_0$ y parámetro $\lambda_0$, mientras que
$ \xi_0'(\cdot),\ T_0',\ \lambda_0' $ representan la dirección de continuación para calcular la siguiente órbita
\cite{Doedel2007}.

Como la variable desconocida es la trayectoria completa $ \xi $ en el intervalo del periodo, se requiere discretizarla.
Para ello se implementa el método de \emph{múltiple disparo} \cite{FARANTOS1998240}. En primer lugar, se define el flujo

\[
    \varphi^{t}(\xi^{(0)}, T, \lambda),
\]

que satisface las siguientes propiedades:

\begin{itemize}
    \item $ \varphi^t:\mathbb{R}^{6} \to \mathbb{R}^{4} $,
    \item $ \varphi^0(\xi^{(0)}, T, \lambda) = \xi^{(0)} $,
    \item $ \varphi^{t+\tau}(\xi^{(0)}, T, \lambda) = \varphi^t(\varphi^{\tau}(\xi^{(0)}, T, \lambda), T, \lambda) $,
    \item $ \dfrac{d}{dt}\varphi^{t}(\xi^{(0)}, T, \lambda) = T F(\varphi^t(\xi^{(0)}, T, \lambda), \lambda; \tilde{\omega}, \tilde{x}, \tilde{k}, K) $.
\end{itemize}

El método consiste en particionar uniformemente el intervalo $[0,1]$ en $N$ subintervalos de tamaño
$\Delta t = 1/N$, y asociar a cada nodo $i$ el vector $\xi^{(i)} = \xi(i \Delta t)$, con $i = 0, \dots, N$.
Cada nodo se considera como una variable independiente y se integra desde él para obtener el flujo
$\varphi^{\Delta t}(\xi^{(i)}, T, \lambda)$. Con ello, se construye el sistema
\begin{equation}
    G(\xi, T, \lambda) =
    \begin{cases}
        \varphi^{\Delta t}(\xi^{(i)}, T, \lambda) - \xi^{(i+1)}, & i = 0, 1, \dots, N-1, \\[1mm]
        \xi^{(N)} - \xi^{(0)},                                                           \\[1mm]
        \frac{\Delta t}{2} \sum_{k \in \{0, N\}} \xi^{(k)^T} \dot{\xi}_{0}^{(k)}
        + \Delta t \sum_{k=2}^{N-1} \xi^{(k)^T} \dot{\xi}_{0}^{(k)} ,
    \end{cases}
    \label{eq:multiple_shooting}
\end{equation}
junto con la ecuación de continuación discretizada
\begin{equation}
    \frac{\Delta t}{2} \sum_{k \in \{0, N\}}
    (\xi^{(k)} - \xi_0^{(k)})^{T} \xi_0'^{(k)} +
    \Delta t \sum_{k=1}^{N-1}
    (\xi^{(k)} - \xi_0^{(k)})^{T} \xi_0'^{(k)}
    + (T - T_0) T_0'
    + (\lambda - \lambda_0) \lambda_0'
    = \Delta s ,
    \label{eq:arclenght_ms}
\end{equation}
lo que da un total de $4N + 5$ ecuaciones (sin incluir la condición de continuación) para un número de variables igual a $4N + 6$.

Para determinar la dirección de continuación
$(\xi_0'(\cdot), T_0', \lambda_0')$, se calcula el espacio nulo del jacobiano completo del sistema extendido $G$, es decir,

\begin{equation}
    G_{(\xi, T, \lambda)}(\xi_0, T_0, \lambda_0)
    \begin{pmatrix}
        \xi_0' \\[1mm] T_0' \\[1mm] \lambda_0'
    \end{pmatrix} = 0.
\end{equation}

De esta manera, la dirección de continuación corresponde a un vector no trivial en el núcleo del Jacobiano
completo $G_{(\xi, T, \lambda)}$, el cual garantiza que las nuevas iteraciones permanezcan sobre la curva de
soluciones periódicas.


Partiendo del dominio permitido para $J_i$ y $\psi_i$, se observa que el sistema
\eqref{eq:multiple_shooting} no satisface los teoremas anteriores para ningún valor de los parámetros,
ya que en un sistema conservativo toda órbita periódica no constituye una solución regular \cite{MUNOZALMARAZ20031}.
En efecto, en cada órbita periódica el sistema \eqref{eq:multiple_shooting} presenta un espacio nulo
de dimensión mayor a uno.

Siguiendo el enfoque propuesto en \cite{MUNOZALMARAZ20031}, es posible introducir un parámetro de perturbación $\lambda$
que permite formular las órbitas periódicas como soluciones regulares. Dicho parámetro, aunque siempre
nulo en la práctica, habilita la continuación con respecto a la cantidad conservada en la órbita, es decir,
la energía. El sistema perturbado se define como
\begin{equation}
    \tilde{G}(\xi, T, \lambda) =
    \begin{cases}
        \tilde{\varphi}^{\Delta t}(\xi^{(i)}, T, \lambda) - \xi^{(i+1)}, & i = 0, 1, ..., N-1 , \\[1mm]
        \xi^{(N)} - \xi^{(0)},                                                                  \\[1mm]
        \frac{\Delta t}{2} \sum_{k \in \{0, N\}} \xi^{(k)^T} \dot{\xi}_{0}^{(k)}
        + \Delta t \sum_{k=2}^{N-1} \xi^{(k)^T} \dot{\xi}_{0}^{(k)},
    \end{cases}
    \label{eq:multiple_shooting_conservative}
\end{equation}
manteniendo la misma ecuación de continuación, pero con un flujo modificado $\tilde{\varphi}^{t}$ cuya
derivada cumple \cite{MUNOZALMARAZ20031}

\[
    \dfrac{d}{dt}\tilde{\varphi}^{t}(\xi_0, T, \lambda)
    = T \left( F(\tilde{\varphi}^t(\xi_0, T, \lambda), p)
    + \lambda \nabla_{\xi} H(\tilde{\varphi}^t(\xi_0, T, \lambda), p) \right).
\]

De esta forma, las órbitas periódicas se convierten en soluciones regulares de
\eqref{eq:multiple_shooting_conservative}, cumpliendo las hipótesis de los teoremas \ref{th:IFT}, \ref{th:Differentiation}
y \ref{th:RegularSolution} para realizar la continuación de soluciones. En consecuencia, el sistema extendido final para continuar órbitas periódicas queda
formulado como
\begin{equation}
    \begin{cases}
        \tilde{\varphi}^{\Delta t}(\xi^{(i)}, T, \lambda) - \xi^{(i+1)} = 0 \qquad i = 0, 1, ..., N-1 , \\[1mm]
        \xi^{(N)} - \xi^{(0)} = 0,                                                                      \\[1mm]
        \frac{\Delta t}{2} \sum_{k \in \{0, N\}} \xi^{(k)^T} \dot{\xi}_{0}^{(k)}
        + \Delta t \sum_{k=2}^{N-1} \xi^{(k)^T} \dot{\xi}_{0}^{(k)} = 0,                                \\[1mm]
        \frac{\Delta t}{2} \sum_{k \in \{0, N\}}
        (\xi^{(k)} - \xi_0^{(k)})^{T} \xi_0'^{(k)} +
        \Delta t \sum_{k=2}^{N-1}
        (\xi^{(k)} - \xi_0^{(k)})^{T} \xi_0'^{(k)}                                                      \\
        + (T - T_0) T_0'
        + (\lambda - \lambda_0) \lambda_0'
        - \Delta s = 0.
    \end{cases}
    \label{eq:Orbit_System}
\end{equation}
Con esta formulación es posible implementar el algoritmo \ref{al:PALC} para realizar la continuación \cite{Meijer2011}.


\subsection{Multiplicadores de Floquet}

Para determinar la estabilidad de las órbitas periódicas y detectar bifurcaciones, es necesario calcular los \emph{multiplicadores de Floquet}.
Estos se obtienen a partir de la \emph{matriz de Floquet}, construida mediante los flujos en cada nodo de la discretización \cite{FloquetMultipliers}.

La matriz de Floquet se calcula como el producto sucesivo de los jacobianos de los flujos en cada nodo:
\begin{equation}
    M = M^{(N-1)} M^{(N-2)} \dots M^{(1)} M^{(0)},
\end{equation}
donde cada bloque \(M^{(k)}\) se define como

\begin{equation}
    M^{(k)}_{ij} = \frac{\partial}{\partial \xi^{(k)}_j} \tilde{\varphi}_{i}^{\Delta t}(\xi^{(k)}, T, \lambda).
\end{equation}

Los multiplicadores de Floquet son los autovalores de \(M\). Para un gran número de nodos, los errores numéricos en el
cálculo de los autovalores se pueden reducir utilizando métodos de factorización estables; en este trabajo se
empleó la \emph{descomposición de Schur} \cite{FloquetMultipliers}.


\subsection{Sistema extendido para la duplicación de periodo}

Para detectar bifurcaciones de duplicación de periodo, se define la siguiente \emph{función de prueba}:
\begin{equation}
    \phi_{PD}(i) = \prod_{i=1}^{4} (\mu_i + 1),
    \label{PDtest}
\end{equation}
la cual cambia de signo al cruzar la bifurcación durante la continuación de la órbita periódica \cite{Meijer2011}.

El sistema extendido para calcular bifurcaciones de duplicación de periodo se formula como
\begin{equation}
    \begin{cases}
        \dfrac{d\xi_{PD}}{dt} = T_{PD} F(\xi_{PD}, \lambda_{PD}; \tilde{\omega}, \tilde{x}, \tilde{k}, K),      \\[1mm]
        \xi_{PD}(0) = \xi_{PD}(1),                                                                              \\[1mm]
        \int_0^1 \xi_{PD}(t)^{T} \dot{\xi}_0(t) \, dt = 0,                                                      \\[1mm]
        \dfrac{dv}{dt} = T_{PD}\, F_{\xi}(\xi_{PD}, \lambda_{PD}; \tilde{\omega}, \tilde{x}, \tilde{k}, K)\, v, \\[1.5mm]
        v(0) = -v(1),                                                                                           \\[1mm]
        \int_0^1 v(t) \cdot v(t)\, dt = 1,
    \end{cases}
    \label{eq:pd_system_general}
\end{equation}
donde $v$ es la eigenfunción definida en \eqref{eq:variationaequation} \cite{Meijer2011}.
Los puntos iniciales $\left(\xi_{PD}^{(0)}(\cdot), \lambda_{PD}^{(0)}, T_{PD}^{(0)}\right)$ corresponden a una
solución de la continuación donde la función de prueba $\phi_{PD}$ cambia de signo, y $v^{(0)}(\cdot)$
es la eigenfunción asociada al multiplicador de Floquet más cercano a $-1$ en ese mismo punto.

Siguiendo la estrategia del \emph{método de múltiple disparo}, la versión discreta del sistema extendido se escribe como
\begin{equation}
    \begin{cases}
        \tilde{\varphi}^{\Delta t}(\xi_{PD}^{(i)}, T_{PD}, \lambda_{PD}) - \xi_{PD}^{(i+1)} = 0, \quad i=0,1,\dots,N-1, \\[1mm]
        \xi_{PD}^{(N)} - \xi_{PD}^{(0)} = 0,                                                                            \\[1mm]
        \frac{\Delta t}{2} \sum_{k \in \{0, N\}} \xi_{PD}^{(k)^T} \dot{\xi}_{0}^{(k)}
        + \Delta t \sum_{k=2}^{N-1} \xi_{PD}^{(k)^T} \dot{\xi}_{0}^{(k)} = 0,                                           \\[1mm]
        \tilde{\Phi}^{\Delta t}(v^{(i)}, T_{PD}, \lambda_{PD}) - v^{(i+1)} = 0, \quad i=0,1,\dots,N-1,                  \\[1mm]
        v^{(N)} + v^{(0)} = 0,                                                                                          \\[1mm]
        \frac{\Delta t}{2} \sum_{k \in \{0, N\}} v^{(k)^T} v^{(k)} +
        \Delta t \sum_{k=2}^{N-1} v^{(k)^T} v^{(k)} = 1,
    \end{cases}
    \label{eq:PD_System}
\end{equation}
donde \(\tilde{\Phi}^t(\cdot)\) representa el flujo perturbado análogo a \(\tilde{\varphi}^t(\cdot)\) aplicado a la eigenfunción $v$, y satisface la ecuación variacional

\[
    \dfrac{d}{dt}\, \tilde{\Phi}^{t}(v^{(0)}, T, \lambda)
    = T \Big( F_\xi(\tilde{\varphi}^t(\xi^{(0)}, T, \lambda), p)
    + \lambda \, \nabla^2_\xi \tilde{H}(\tilde{\varphi}^t(\xi^{(0)}, T, \lambda), p) \Big) \,
    \tilde{\Phi}^{t}(v^{(0)}, T, \lambda).
\]

\subsection{Transición a la rama de doble periodo}

Una vez detectada la bifurcación de duplicación de periodo
\((\xi_{PD}(\cdot), \lambda_{PD}, T_{PD})\), se puede continuar hacia la rama emergente de doble periodo
duplicando el periodo y seleccionando la dirección correcta para la continuación.

\paragraph{Duplicación del periodo:}
Duplicar el periodo implica establecer \(T = 2\,T_{PD}\) y, de manera consistente, duplicar el número de nodos en la discretización.
Esto conduce a \(\Delta t = 1/(2N)\) y \(i = 0, 1, \dots, 2N\), con
\begin{equation}
    \xi^{(i)} =
    \begin{cases}
        \xi_{PD}^{(i)} ,  & i \leq N, \\[1mm]
        \xi_{PD}^{(i-N)}, & i > N,
    \end{cases}
    \label{eq:doubling_nodes}
\end{equation}
para reconstruir la órbita inicial de doble periodo.

\paragraph{Dirección de la rama emergente:}
La dirección de la nueva rama se obtiene a partir de la eigenfunción \(v(\cdot)\) en la bifurcación:
\begin{equation}
    \xi_0'^{(i)} =
    \begin{cases}
        v^{(i)},    & i \leq N, \\[1mm]
        -v^{(i-N)}, & i > N,
    \end{cases}
    \label{eq:2T_direction}
\end{equation}
mientras que \(\lambda_0' = 0\) y \(T_0' = 0\) \cite{Meijer2011}.

\subsection{Continuación de doblamientos de periodo}

El sistema \eqref{eq:PD_System}, de manera análoga a las bifurcaciones de equilibrio, también cumple con las hipótesis de los teoremas
\ref{th:IFT}, \ref{th:Differentiation} y \ref{th:RegularSolution} \cite{Meijer2011}. Por lo tanto, es posible realizar la continuación numérica de este tipo de bifurcación.

Para ello, se introduce la ecuación de proyección
\begin{align}
    \Delta s = & \frac{\Delta t}{2} \sum_{k \in \{0, N\}}
    (\xi^{(k)} - \xi_0^{(k)})^{T} \xi_0'^{(k)}
    + \Delta t \sum_{k=1}^{N-1} (\xi^{(k)} - \xi_0^{(k)})^{T} \xi_0'^{(k)} \nonumber \\
               & \quad + \frac{\Delta t}{2} \sum_{k \in \{0, N\}}
    (v^{(k)} - v_0^{(k)})^{T} v_0'^{(k)}
    + \Delta t \sum_{k=1}^{N-1} (v^{(k)} - v_0^{(k)})^{T} v_0'^{(k)} \nonumber       \\
               & \quad + (T - T_0) T_0'
    + (\lambda - \lambda_0) \lambda_0'
    + (\mu - \mu_0) \mu_0'
\end{align}
donde \(\mu\) representa un segundo parámetro del sistema (por ejemplo, \(\tilde{\omega}\), \(\tilde{k}\) o \(\tilde{x}\)).

Con esta formulación, se puede aplicar directamente el algoritmo \ref{al:PALC} para llevar a
cabo la continuación de la bifurcación de doblamiento de periodo.

\section{Sección de Poincaré}

Para visualizar la dinámica del sistema en el espacio de fases, en este trabajo se selecciona el plano
\((\psi_2, J_2)\), de modo que el análisis se reduce a la intersección de una hipersuperficie de energía \(H = E\)
con la superficie \(\psi_1 = 0\). En esta intersección pueden coexistir varias ramas de energía que cumplen
\(H = E\), por lo que la dinámica se restringe a los puntos donde la velocidad de cruce en \(J_1\) es positiva
y la curvatura de la energía respecto a \(J_1\) satisface una determinada condición de concavidad \cite{mossmann2006semiclassical}.

Formalmente, las condiciones utilizadas para construir la sección de Poincaré de interés son:

\begin{equation}
    \begin{aligned}
         & \psi_1 \in \{0, \pi\},                                      \\
         & \tilde{H}(\psi_1, \psi_2, J_1, J_2) = \tilde{E},            \\
         & \frac{\partial \tilde{H}}{\partial J_1} = \dot{\psi}_1 > 0, \\
         & \frac{\partial^2 \tilde{H}}{\partial J_1^2} > 0.
    \end{aligned}
\end{equation}

De manera explícita:
\begin{enumerate}
    \item La superficie de Poincaré se toma en \(\psi_1 = 0\) o en \(\psi_1 = \pi\), dependiendo de la situación.
    \item Se restringe a la hipersuperficie de energía \(\tilde{H} = \tilde{E}\).
    \item Se consideran solo los puntos donde la velocidad de cruce de \(J_1\) es positiva: \(\dot{\psi}_1 > 0\).
    \item Se seleccionan los puntos donde la energía es localmente convexa respecto a \(J_1\): \(\partial^2 \tilde{H} / \partial J_1^2 > 0\).
\end{enumerate}

En esta sección, la dinámica del sistema puede visualizarse mediante distintos tipos de comportamiento en el espacio de fases:

\begin{itemize}
    \item \emph{Toros quasi-periódicos:} conjuntos de puntos que se acumulan a lo largo de curvas cerradas.
    \item \emph{Órbitas periódicas:} puntos fijos o centrales alrededor de los toros.
    \item \emph{Zonas caóticas:} regiones donde los puntos no muestran una estructura aparente.
\end{itemize}

La sección de Poincaré implementada en este trabajo se utiliza para localizar órbitas periódicas que sirvan como condiciones iniciales
para la continuación. El procedimiento geométrico seguido es el siguiente:

\begin{enumerate}
    \item Se identifica un toro que contiene una órbita periódica cercana.
    \item Se aproxima la órbita periódica mediante el \emph{punto promedio} del conjunto de puntos del toro.
    \item Se genera un nuevo toro centrado en este punto promedio, más cercano a la órbita periódica real.
    \item Se repite el proceso hasta que la desviación estándar del conjunto de puntos del toro sea suficientemente pequeña,
          garantizando así una aproximación precisa de la órbita periódica.
\end{enumerate}

De esta manera, se obtiene una estimación confiable de la órbita periódica, que puede emplearse como semilla inicial
para métodos de continuación numérica.

Con el objetivo de visualizar las secciones de Poincaré de forma cerrada, se hace una transformación a
coordenadas esféricas siguiendo las relaciones
\begin{equation}
    \begin{aligned}
        X & = \dfrac{2 \sqrt{J_2 (K - J_2)} \cos(\psi_2)}{K}, \\
        Y & = \dfrac{2 \sqrt{J_2 (K - J_2)} \sin(\psi_2)}{K}, \\
        Z & = \dfrac{2 J_2 - K}{K}.
    \end{aligned}
    \label{eq:spherical_coordinates}
\end{equation}

Geométricamente, esta transformación mapea la sección de Poincaré en una esfera unitaria
donde los polos corresponden a los valores extremos de \(J_2\) (es decir, \(J_2 = 0\) y \(J_2 = K\)),
y el ecuador representa el valor intermedio \(J_2 = K/2\). Esta representación facilita la visualización
y el análisis de la dinámica del sistema en el espacio de fases.







